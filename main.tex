%%%%%%%%%%%%%%%%%
% This is an sample CV template created using altacv.cls
% (v1.3, 10 May 2020) written by LianTze Lim (liantze@gmail.com). Now compiles with pdfLaTeX, XeLaTeX and LuaLaTeX.
% (v1.6.5b, 27 Jun 2023) forked by Nicolás Omar González Passerino (nicolas.passerino@gmail.com)
%
%% It may be distributed and/or modified under the
%% conditions of the LaTeX Project Public License, either version 1.3
%% of this license or (at your option) any later version.
%% The latest version of this license is in
%%    http://www.latex-project.org/lppl.txt
%% and version 1.3 or later is part of all distributions of LaTeX
%% version 2003/12/01 or later.
%%%%%%%%%%%%%%%%

%% If you need to pass whatever options to xcolor
\PassOptionsToPackage{dvipsnames}{xcolor}

%% If you are using \orcid or academicons
%% icons, make sure you have the academicons
%% option here, and compile with XeLaTeX
%% or LuaLaTeX.
% \documentclass[10pt,a4paper,academicons]{altacv}

%% Use the "normalphoto" option if you want a normal photo instead of cropped to a circle
% \documentclass[10pt,a4paper,normalphoto]{altacv}

%% Fork (before v1.6.5a): CV dark mode toggle enabler to use a inverted color palette.
%% Use the "darkmode" option if you want a color palette used to 
% \documentclass[10pt,a4paper,ragged2e,withhyper,darkmode]{altacv}

\documentclass[10pt,a4paper,ragged2e,withhyper]{altacv}

%% AltaCV uses the fontawesome5 and academicons fonts
%% and packages.
%% See http://texdoc.net/pkg/fontawesome5 and http://texdoc.net/pkg/academicons for full list of symbols. You MUST compile with XeLaTeX or LuaLaTeX if you want to use academicons.

% Change the page layout if you need to
\geometry{left=1.2cm,right=1.2cm,top=1cm,bottom=1cm,columnsep=0.75cm}

% The paracol package lets you typeset columns of text in parallel
\usepackage{paracol}

% Change the font if you want to, depending on whether
% you're using pdflatex or xelatex/lualatex
\ifxetexorluatex
  % If using xelatex or lualatex:
  \setmainfont{Roboto Slab}
  \setsansfont{Lato}
  \renewcommand{\familydefault}{\sfdefault}
\else
  % If using pdflatex:
  \usepackage[rm]{roboto}
  \usepackage[defaultsans]{lato}
  % \usepackage{sourcesanspro}
  \renewcommand{\familydefault}{\sfdefault}
\fi

% Fork (before v1.6.5a): Change the color codes to test your personal variant on any mode
\ifdarkmode%
  \definecolor{PrimaryColor}{HTML}{C69749}
  \definecolor{SecondaryColor}{HTML}{D49B54}
  \definecolor{ThirdColor}{HTML}{1877E8}
  \definecolor{BodyColor}{HTML}{ABABAB}
  \definecolor{EmphasisColor}{HTML}{ABABAB}
  \definecolor{BackgroundColor}{HTML}{191919}
\else%
  \definecolor{PrimaryColor}{HTML}{001F5A}
  \definecolor{SecondaryColor}{HTML}{0039AC}
  \definecolor{ThirdColor}{HTML}{F3890B}
  \definecolor{BodyColor}{HTML}{666666}
  \definecolor{EmphasisColor}{HTML}{2E2E2E}
  \definecolor{BackgroundColor}{HTML}{E2E2E2}
\fi%

\colorlet{name}{PrimaryColor}
\colorlet{tagline}{SecondaryColor}
\colorlet{heading}{PrimaryColor}
\colorlet{headingrule}{ThirdColor}
\colorlet{subheading}{SecondaryColor}
\colorlet{accent}{SecondaryColor}
\colorlet{emphasis}{EmphasisColor}
\colorlet{body}{BodyColor}
\pagecolor{BackgroundColor}

% Change some fonts, if necessary
\renewcommand{\namefont}{\Huge\rmfamily\bfseries}
\renewcommand{\personalinfofont}{\small\bfseries}
\renewcommand{\cvsectionfont}{\LARGE\rmfamily\bfseries}
\renewcommand{\cvsubsectionfont}{\large\bfseries}

% Change the bullets for itemize and rating marker
% for \cvskill if you want to
\renewcommand{\itemmarker}{{\small\textbullet}}
\renewcommand{\ratingmarker}{\faCircle}

\NewInfoField{twitter}{\faTwitter}[https://twitter.com/]

%% sample.bib contains your publications
%% \addbibresource{main.bib}

\begin{document}
    \name{Нурлат Бекдуллаев}
    % \tagline{Python Developer}
    %% You can add multiple photos on the left or right
    \photoL{4cm}{IMG_0473}

    \personalinfo{
        \email{nurlat.bekdullayev@alumni.nu.edu.kz}\smallskip
        \phone{+7-706-478-0075}
        \location{Kazakhstan}\\
        \github{natuspati}
        \linkedin{nurlat}
        \twitter{natuspati}
        %\homepage{nicolasomar.me}
        %\medium{nicolasomar}
        %% You MUST add the academicons option to \documentclass, then compile with LuaLaTeX or XeLaTeX, if you want to use \orcid or other academicons commands.
        % \orcid{0000-0000-0000-0000}
        %% You can add your own arbtrary detail with
        %% \printinfo{symbol}{detail}[optional hyperlink prefix]
        % \printinfo{\faPaw}{Hey ho!}[https://example.com/]
        %% Or you can declare your own field with
        %% \NewInfoFiled{fieldname}{symbol}[optional hyperlink prefix] and use it:
        % \NewInfoField{gitlab}{\faGitlab}[https://gitlab.com/]
        % \gitlab{your_id}
    }

    \makecvheader
    %% Depending on your tastes, you may want to make fonts of itemize environments slightly smaller
    % \AtBeginEnvironment{itemize}{\small}

    %% Set the left/right column width ratio to 6:4.
    \columnratio{0.25}

    % Start a 2-column paracol. Both the left and right columns will automatically
    % break across pages if things get too long.
    \begin{paracol}{2}
        % ----- TECH STACK -----
        \cvsection{ТЕХ. СТЭК}
            \cvtag{Django}
            \cvtag{FastAPI}

            \cvtag{Django REST Framework}
            \cvtag{Celery}
            \cvtag{Pytest}
            \cvtag{Asyncio}
            \cvtag{PyTorch}
            \cvtag{openCV}
            \medskip

            \cvtag{Docker}
            \cvtag{PostgreSQL}
            \cvtag{AWS}
            \cvtag{MongoDB}
            \cvtag{Git}

            \cvtag{Github Actions}
            \medskip

            \cvtag{React}
            \cvtag{ElasticUI}
            \medskip

            \cvtag{Python}
            \cvtag{SQL}
            \cvtag{Matlab}
            \cvtag{Javascript}

        % ----- TECH STACK -----

        % ----- PRINCIPLES -----
        \cvsection{ПРИНЦИПЫ}
            \cvtag{PEP 8}
            \cvtag{TDD}
            \cvtag{BDD}
            \cvtag{DRY}
            \cvtag{SOLID}
            \cvtag{OOP}
            \cvtag{Agile}
            \cvtag{Scrum}

            %% Yeah I didn't spend too much time making all the
            %% spacing consistent... sorry. Use \smallskip, \medskip,
            %% \bigskip, \vpsace etc to make ajustments.
        % ----- PRINCIPLES -----

        % ----- LANGUAGES -----
        \cvsection{ЯЗЫКИ}
            \cvlang{Английский}{C1, IELTS 8.0}

            \cvlang{Казахский}{Родной}

            \cvlang{Русский}{Свободный}
            %% Yeah I didn't spend too much time making all the
            %% spacing consistent... sorry. Use \smallskip, \medskip,
            %% \bigskip, \vpsace etc to make ajustments.
        % ----- LANGUAGES -----

        % ----- REFERENCES -----
        \cvsection{Сертификаты}
            \withglobe{Advanced Django: Mastering Django and Django Rest Framework Specialization}{\href{https://www.coursera.org/account/accomplishments/specialization/certificate/25VSHAPPR553}{Coursera}}{}
            \divider

            \withglobe{Django for Everybody}{\href{https://www.coursera.org/account/accomplishments/specialization/certificate/MDYWCZU7WH52}{University of Michigan}}{}
            \divider

            \withglobe{Containerized Applications on AWS}{\href{https://www.coursera.org/account/accomplishments/certificate/KNQTAQ6UHYZG}{AWS Training}}{}
            \divider
        % ----- REFERENCES -----

        % ----- MOST PROUD -----
        % \cvsection{Most Proud of}

        % \cvachievement{\faTrophy}{Fantastic Achievement}{and some details about it}\\
        % \divider
        % \cvachievement{\faHeartbeat}{Another achievement}{more details about it of course}\\
        % \divider
        % \cvachievement{\faHeartbeat}{Another achievement}{more details about it of course}
        % ----- MOST PROUD -----

        % \cvsection{A Day of My Life}

        % Adapted from @Jake's answer from http://tex.stackexchange.com/a/82729/226
        % \wheelchart{outer radius}{inner radius}{
        % comma-separated list of value/text width/color/detail}
        % \wheelchart{1.5cm}{0.5cm}{%
        %   6/8em/accent!30/{Sleep,\\beautiful sleep},
        %   3/8em/accent!40/Hopeful novelist by night,
        %   8/8em/accent!60/Daytime job,
        %   2/10em/accent/Sports and relaxation,
        %   5/6em/accent!20/Spending time with family
        % }

        % use ONLY \newpage if you want to force a page break for
        % ONLY the current column
        \newpage

        %% Switch to the right column. This will now automatically move to the second
        %% page if the content is too long.
        \switchcolumn

        % ----- ABOUT ME -----
        % \cvsection{Обо мне}
        %     \begin{quote}
        %         Back-end разработчик, который начал само обучаться в 2022. Бывший докторант Вычислительной Механики.
        %     \end{quote}
        % ----- ABOUT ME -----

        % ----- PROJECTS -----
        \cvsection{Фриланс}
            % \cvComProject{User management integration app}
            \cvevent{Интеграция IAM пользователей}{Sonerep}{June 2023}{}
            \begin{itemize}
                \item Создал Django плагин для аутентификации и контроля доступа
                \item Интегрирует аутентификацию и права с Keycloak используя SAML 2 протокол
            \end{itemize}

        \cvsection{Проекты}
            \cvtest{Phrosty - on-demand услуга уборок}{\cvreference{\faGithub}{https://github.com/natuspati/phrosty}}{}{}
            \begin{itemize}
                \item Разработан используя TDD (Pytest), асинхронный FastAPI с чистым SQL для макс. производительности
            \end{itemize}
            \medskip

            \cvtest{Django Chat - асинхронный чат}{\cvreference{\faGithub}{https://github.com/natuspati/django-chat}}{}{}
            \begin{itemize}
                \item Использует Django 4.2 async ORM, Webscokets и React стэк
            \end{itemize}
            \medskip

            \cvtest{Country Guess Game - отгадай скрытую страну}{\cvreference{\faGithub}{https://github.com/natuspati/GuessCountryGame}}{}{}
            \begin{itemize}
                \item Только Celery и Django Sessions используются для аутентификации и данных пользователей
                \item CI/CD пайплайн используя Docker, Github Actions и AWS EC2/ECR
            \end{itemize}
        % ----- PROJECTS -----

        % ----- EXPERIENCE -----
        \cvsection{Опыт}
            \cvevent{Инженер-исследователь}{\href{https://vccimaging.org/}{Computational Imaging Group}}{Сен 2022 -- Май 2023}{Тувал, Саудовская Арабия}
            \begin{itemize}
                \item Разработал модели для калибровки камер с помошью PyTocrh
                \item Создал часть data pipline: data storage,  transformation and processing для обучения моделей
                \item Расширил модуль калибровки openCV с разработанной моделью
            \end{itemize}
            \divider

            \cvevent{Инженер-исследователь}{\href{https://composites.kaust.edu.sa/}{Composite Material Simulation Laboratory}}{Июль 2021 -- Aвг 2022}{Тувал, Саудовская Арабия}
            \begin{itemize}
                \item Интегрировал библиотеку Python-openCV с мульти-камерной системой
                \item Разработал систему для измерений механического урона с помощью камер и зеркал
            \end{itemize}
        % ----- EXPERIENCE -----

        % ----- EDUCATION -----
        \cvsection{Образование}
            \cvevent{Магистр Вычислительной Механики}{\href{https://www.kaust.edu.sa/en/}{Университет Науки и Технологии имени короля Абдуллы}}{Сен 2019 -- Июнь 2021}{Тувал, Саудовская Арабия}

            \divider

            \cvevent{Бакалавр Механической Инженерии}{\href{https://nu.edu.kz/}{Назарбаев Университет}}{Сен 2015 -- Июнь 2019}{Астана, Казахстан}
            % \begin{itemize}
            %     \item GPA: 3.61
            % \end{itemize}
        % ----- EDUCATION -----
        
        
    \end{paracol}
\end{document}